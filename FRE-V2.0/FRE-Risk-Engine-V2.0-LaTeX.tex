\documentclass[12pt]{article}

\usepackage{amsmath, amssymb}
\usepackage{geometry}
\usepackage{setspace}
\usepackage{lmodern}
\geometry{margin=1in}

\begin{document}

% ---------------------------------------------------------
% TITLE PAGE (ONLY ONE — FIXED)
% ---------------------------------------------------------
\begin{titlepage}
    \thispagestyle{empty}
    \centering
    \vspace*{2cm}

    {\Large \textbf{Flexionization Risk Engine (FRE)}}\\[10pt]
    {\large \textbf{Version 2.0}}\\[22pt]

    {\large \textbf{Formal Specification and Structural Dynamics Model}}\\[35pt]

    {\large Maryan Bogdanov}\\[6pt]
    {\normalsize Independent Researcher}\\[4pt]
    {\normalsize \texttt{m7823445@gmail.com}}\\[28pt]

    {\normalsize \textbf{Date:} 2025}\\[4pt]
    {\normalsize \textbf{Document Type:} Research / Technical Specification}\\[4pt]
    {\normalsize \textbf{LaTeX Source:} FRE-Risk-Engine-V2.0-LaTeX.tex}\\[4pt]
    {\normalsize \textbf{DOI:} (assigned after Zenodo publication)}\\[32pt]

    \vfill

    \begin{spacing}{1}
    {\small
        This document presents the full mathematical specification of the
        Flexionization Risk Engine (FRE) Version 2.0, including axioms,
        multidimensional deviation model, equilibrium dynamics, corrective
        operators, stability guarantees, and system geometry. It is intended
        for academic review, formal archival publication, and application in
        real-world risk-engineering systems.
    }
    \end{spacing}

    \vfill
\end{titlepage}

% ---------------------------------------------------------
% ABSTRACT (ONLY ONE — FIXED)
% ---------------------------------------------------------
\begin{abstract}
The Flexionization Risk Engine (FRE) Version 2.0 is a multidimensional,
structurally complete risk–control framework based entirely on internal
system dynamics rather than market-based triggers. Unlike traditional
risk engines that rely on volatility spikes, heuristic thresholds, or
external price movements, FRE defines stability through a vector deviation
model
\[
    \vec{\Delta} = (\Delta_m, \Delta_L, \Delta_H, \Delta_R, \Delta_C)
    \in \mathbb{R}^5,
\]
representing margin, exposure, liquidity, risk–parameter, and capital
deviations of the full state \( X \).

Corrective behavior is governed by the multidimensional equilibrium
indicator \( \mathrm{FXI} \) and a vector corrective operator \( \vec{E} \),
which maps the system toward structural equilibrium under bounded,
continuous transitions. FRE 2.0 provides formal guarantees of stability,
convergence, admissibility, and geometric contraction, establishing a
unique equilibrium attractor independent of market conditions.

This document presents the complete mathematical specification of FRE 2.0,
including the axiomatic foundation, deviation geometry, equilibrium dynamics,
system evolution laws, stability theorems, critical edge scenarios, and the
simulation framework required for practical implementation. The model serves
as a unified structural basis for risk engines across CeFi, DeFi, banking,
derivatives, and automated hedging systems.
\end{abstract}

% ---------------------------------------------------------
% KEYWORDS
% ---------------------------------------------------------
\noindent\textbf{Keywords:}
Flexionization, Risk Engine, Structural Dynamics, Equilibrium Model,
Multidimensional Deviation, Stability Theory, CeFi Risk Control,
DeFi Risk Architecture, Contraction Mapping, Nonlinear Systems.

% ---------------------------------------------------------
% CONTENTS
% ---------------------------------------------------------
\tableofcontents
\newpage

\section{Introduction}

The Flexionization Risk Engine (FRE) is a structural, fully deterministic
risk–control framework designed to operate independently of external
market signals, volatility spikes, or heuristic trigger mechanisms.
Traditional risk engines across CeFi, DeFi, banking, derivatives, and
automated hedging systems rely on discontinuous adjustments driven by
market behavior. These mechanisms introduce procyclicality, instability,
and systemic amplification during stress events.

FRE Version~2.0 replaces these approaches with a mathematically rigorous,
multidimensional equilibrium model grounded exclusively in the internal
state of the system. The core concept is the structural deviation vector
\[
    \vec{\Delta} = (\Delta_m, \Delta_L, \Delta_H,
                    \Delta_R, \Delta_C) \in \mathbb{R}^5,
\]
which captures the state of margins, exposures, hedging, risk parameters,
and capital buffers.

The corrective dynamics of the system are governed by the structural
equilibrium indicator $\mathrm{FXI}$ and a vector corrective operator
$\vec{E}$ that ensures continuous, bounded, and contracting transitions.
These dynamics guarantee convergence toward a unique structural
equilibrium attractor, independent of price movements or external
volatility.

FRE 2.0 provides a unified mathematical foundation for designing
risk–control architectures across financial and computational domains.
The model includes:

\begin{itemize}
    \item multidimensional deviation geometry,
    \item axiomatic consistency requirements,
    \item vector equilibrium dynamics,
    \item contraction-based stability guarantees,
    \item critical edge–case analysis,
    \item and a simulation framework for applied systems.
\end{itemize}

This specification defines the complete formal structure of FRE
Version~2.0, intended for academic review, research usage, protocol
design, and integration into real-world risk–engine architectures.

\section{Axioms}

FRE Version~2.0 is built on a set of structural axioms that define the
properties of the multidimensional deviation vector, the equilibrium
indicator, and the corrective dynamics. These axioms ensure mathematical
consistency, boundedness, continuity, and convergence of the system under
all admissible transitions.

\subsection{Axiom 1: Multidimensional State}

The system state is defined as a vector
\[
    X = (m, L, H, R, C),
\]
representing margin configuration, limit and exposure structure, hedging
and liquidity state, risk–parameter configuration, and capital buffers.
All components must be internally computable and continuously
differentiable almost everywhere.

\subsection{Axiom 2: Structural Deviation Vector}

The structural deviation is given by the mapping
\[
    \vec{\Delta} = D(X)
    = \bigl(
        D_m(X), D_L(X), D_H(X),
        D_R(X), D_C(X)
      \bigr)
    \in \mathbb{R}^5,
\]
where each deviation component is continuous, bounded, and monotonic with
respect to its corresponding subsystem.

\subsection{Axiom 3: Equilibrium Indicator}

The structural equilibrium indicator is a scalar function
\[
    \mathrm{FXI} = F(\vec{\Delta}),
\]
defined on $\mathbb{R}^5$, strictly monotonic in each deviation component,
and satisfying:
\[
    \mathrm{FXI} > 1 \;\Rightarrow\; \text{expanded state}, \qquad
    \mathrm{FXI} < 1 \;\Rightarrow\; \text{compressed state}, \qquad
    \mathrm{FXI} = 1 \;\Rightarrow\; \text{structural symmetry}.
\]

\subsection{Axiom 4: Corrective Operator}

The corrective operator is a vector mapping
\[
    \vec{E}: \mathbb{R}^5 \rightarrow \mathbb{R}^5,
\]
which determines the target equilibrium deviation for the next state. The
operator must be:

\begin{itemize}
    \item total (defined for all admissible deviation vectors),
    \item continuous in every component,
    \item bounded in magnitude,
    \item monotonic relative to the equilibrium indicator $F$,
    \item contracting around equilibrium.
\end{itemize}

\subsection{Axiom 5: Bounded Transitions}

For all admissible transitions, the update
\[
    X_{t+1} = X_t + C_t
\]
must satisfy
\[
    \| C_t \| \leq L,
\]
for a finite structural bound $L > 0$, ensuring the absence of
discontinuous structural jumps.

\subsection{Axiom 6: Dynamic Consistency}

The equilibrium indicator must satisfy
\[
    F\bigl(D(X_{t+1})\bigr)
    =
    F(\vec{\Delta}_{t+1})
    =
    F(\vec{E}(\vec{\Delta}_t)),
\]
ensuring that the next state is consistent with the prescribed corrective
dynamics.

\subsection{Axiom 7: Existence of a Corrective Step}

For every admissible state $X_t$, there must exist at least one bounded
correction $C_t$ such that
\[
    \vec{\Delta}_{t+1}
    =
    \vec{E}(\vec{\Delta}_t),
\]
ensuring that structural equilibrium is always reachable.

\subsection{Axiom 8: Bounded Equilibrium Dynamics}

The equilibrium indicator must remain bounded:
\[
    \mathrm{FXI} \leq M,
\]
for a fixed constant $M > 1$, preventing unbounded structural drift.

These axioms together define a deterministic, continuous, and
contraction-driven structural system that forms the foundation of FRE
Version~2.0.

\section{Formal Model}

The formal model of FRE Version~2.0 defines the full mathematical
structure of system evolution, including multidimensional deviation,
equilibrium evaluation, corrective dynamics, and admissible transitions.
The model provides a deterministic update rule that ensures continuous,
bounded, and contracting structural adjustments.

\subsection{State Representation}

The internal state of the system is represented as
\[
    X = (m, L, H, R, C),
\]
where each component corresponds to a structural subsystem:

\begin{itemize}
    \item $m$ — margin configuration,
    \item $L$ — limits and exposures,
    \item $H$ — hedging and liquidity structure,
    \item $R$ — risk–parameter configuration,
    \item $C$ — capital buffers and reserves.
\end{itemize}

All components must be internally computable from on-chain or off-chain
system metadata and represent the complete structural footprint of the
system.

\subsection{Deviation Mapping}

The deviation operator maps a system state into its 5-dimensional deviation
vector:
\[
    \vec{\Delta}
    =
    D(X)
    =
    \bigl(
        D_m(X),\,
        D_L(X),\,
        D_H(X),\,
        D_R(X),\,
        D_C(X)
    \bigr)
    \in \mathbb{R}^5.
\]

Each component of the deviation must satisfy:
\begin{itemize}
    \item continuity,
    \item boundedness,
    \item monotonicity with respect to subsystem imbalance,
    \item admissibility for all possible system states.
\end{itemize}

\subsection{Equilibrium Indicator}

The scalar equilibrium indicator is defined as a continuous mapping
\[
    \mathrm{FXI} = F(\vec{\Delta}),
\]
where $F:\mathbb{R}^5 \to \mathbb{R}^+$ is strictly monotonic and satisfies:

\[
    F(\vec{\Delta}) = 1 
    \iff 
    \text{system in structural symmetry},
\]

\[
    F(\vec{\Delta}) > 1 
    \iff 
    \text{expanded or destabilized structural state},
\]

\[
    F(\vec{\Delta}) < 1 
    \iff 
    \text{compressed or overly conservative structural state}.
\]

\subsection{Corrective Operator}

Corrective adjustments are governed by the vector operator
\[
    \vec{E}: \mathbb{R}^5 \rightarrow \mathbb{R}^5,
\]
which prescribes the target structural deviation for the next moment:
\[
    \vec{\Delta}_{t+1}
    =
    \vec{E}(\vec{\Delta}_t).
\]

The operator $\vec{E}$ must be:
\begin{itemize}
    \item continuous,
    \item bounded,
    \item structurally monotonic,
    \item equilibrium–seeking,
    \item contracting in a neighborhood of equilibrium.
\end{itemize}

\subsection{FXI-Based Dynamics}

The equilibrium indicator determines the contraction direction of the
system:

\[
    F(\vec{\Delta}_t) > 1
    \;\Rightarrow\;
    \vec{E}(\vec{\Delta}_t) < \vec{\Delta}_t,
\]

\[
    F(\vec{\Delta}_t) < 1
    \;\Rightarrow\;
    \vec{E}(\vec{\Delta}_t) > \vec{\Delta}_t,
\]

\[
    F(\vec{\Delta}_t) = 1
    \;\Rightarrow\;
    \vec{E}(\vec{\Delta}_t) = \vec{\Delta}_t.
\]

This guarantees motion toward structural equilibrium.

\subsection{System Update Rule}

The system evolves via the state transition:
\[
    X_{t+1}
    =
    X_t + C_t,
\]
where $C_t$ is a bounded structural correction satisfying:

\[
    D(X_{t+1}) = \vec{E}(D(X_t)).
\]

Thus, the deviation evolution is:
\[
    \vec{\Delta}_{t+1}
    =
    \vec{E}(\vec{\Delta}_t).
\]

\subsection{Admissible Corrections}

For every state $X_t$, there must exist at least one bounded correction
$C_t$ such that:

\[
    \|C_t\| \leq L,
\]
\[
    D(X_t + C_t) = \vec{E}(D(X_t)).
\]

This ensures that structural symmetry is always reachable through
continuous transitions.

\subsection{Equilibrium Point}

A state $X^\ast$ is an equilibrium if:
\[
    D(X^\ast)
    =
    \vec{E}(D(X^\ast)).
\]

Under the contraction property of $\vec{E}$, such an equilibrium is:

\begin{itemize}
    \item unique,
    \item globally stable,
    \item reachable from any admissible initial state.
\end{itemize}

This concludes the formal mathematical definition of FRE 2.0.

\section{Dynamics}

The dynamics of FRE Version~2.0 describe how the system evolves under
the multidimensional corrective operator. The update process is fully
deterministic, continuous, bounded, and monotonic with respect to the
equilibrium indicator.

\subsection{Deviation Evolution}

The core dynamic rule of FRE is the evolution of the deviation vector:
\[
    \vec{\Delta}_{t+1}
    =
    \vec{E}(\vec{\Delta}_t),
\]
where $\vec{E}$ is the multidimensional corrective operator defined in
Section~3. The operator determines the structural direction of motion and
ensures contraction toward equilibrium.

The deviation evolution satisfies:

\begin{itemize}
    \item if $F(\vec{\Delta}_t) > 1$ then
    \[
        \vec{E}(\vec{\Delta}_t) < \vec{\Delta}_t,
    \]
    \item if $F(\vec{\Delta}_t) < 1$ then
    \[
        \vec{E}(\vec{\Delta}_t) > \vec{\Delta}_t,
    \]
    \item if $F(\vec{\Delta}_t) = 1$ then
    \[
        \vec{E}(\vec{\Delta}_t) = \vec{\Delta}_t.
    \]
\end{itemize}

Thus, the sign of the equilibrium indicator determines the contraction or
expansion direction of the deviation vector.

\subsection{State Evolution}

The system state evolves through bounded structural corrections:
\[
    X_{t+1}
    =
    X_t + C_t,
\]
where $C_t$ is an admissible structural correction satisfying:
\[
    D(X_{t+1})
    =
    \vec{E}(D(X_t)).
\]

Because the correction is bounded:
\[
    \|C_t\| \leq L,
\]
the system moves continuously without discontinuous jumps or heuristic
override events.

\subsection{Geometric Contraction}

A key property of FRE is geometric contraction toward equilibrium. There
exists a constant $0 < k < 1$ such that:
\[
    \|\vec{\Delta}_{t+1} - \vec{\Delta}^\ast\|
    \leq
    k \cdot
    \|\vec{\Delta}_{t} - \vec{\Delta}^\ast\|,
\]
where $\vec{\Delta}^\ast$ is the equilibrium deviation.

This implies:

\begin{itemize}
    \item convergence is exponential in discrete time,
    \item equilibrium is globally stable,
    \item deviation cannot oscillate or diverge,
    \item the system is resistant to internal structural shocks.
\end{itemize}

\subsection{Admissible Trajectories}

A deviation trajectory
\[
    \{ \vec{\Delta}_0, \vec{\Delta}_1, \vec{\Delta}_2, \dots \}
\]
is admissible if:

\begin{enumerate}
    \item $\vec{\Delta}_{t+1} = \vec{E}(\vec{\Delta}_t)$ for all $t$,
    \item $\vec{\Delta}_t$ remains within the admissible domain
          $\mathcal{D} \subset \mathbb{R}^5$,
    \item $\mathrm{FXI}( \vec{\Delta}_t )$ is bounded for all $t$,
    \item the corresponding corrections $C_t$ are bounded.
\end{enumerate}

These conditions ensure that every deviation trajectory reflects a
structurally feasible evolution of the system.

\subsection{Structural Symmetry}

The system reaches structural symmetry when:
\[
    F(\vec{\Delta}_t) = 1,
\]
in which case:
\[
    \vec{\Delta}_{t+1} = \vec{\Delta}_{t}.
\]

This means the corrective operator leaves the state invariant, and the
system remains in equilibrium without further structural movement.

\subsection{Non-Equilibrium Motion}

For non-equilibrium states, the system moves strictly toward equilibrium:
\[
    \vec{\Delta}_{t+1}
    \neq
    \vec{\Delta}_t,
\]
and the magnitude of motion satisfies:
\[
    0
    <
    \|\vec{\Delta}_{t+1} - \vec{\Delta}_t\|
    \leq
    \|\vec{\Delta}_t - \vec{\Delta}^\ast\|.
\]

Thus:

\begin{itemize}
    \item the system cannot stagnate outside equilibrium,
    \item the system cannot reverse direction,
    \item the system cannot leave the equilibrium basin.
\end{itemize}

\subsection{Continuous Structural Geometry}

Although FRE evolves in discrete steps, the structural movement obeys a
continuous geometric path in the deviation space. The sequence
$\{\vec{\Delta}_t\}$ forms a contraction mapping toward a unique fixed
point, generating a smooth structural trajectory in $\mathbb{R}^5$.

This continuous geometry is essential for stability, predictability, and
risk–engine designs that must avoid discontinuous liquidation or
volatility-induced shocks.

\section{Stability Theorems}

The stability of FRE Version~2.0 follows from the contraction properties
of the corrective operator and the continuity of the deviation mapping.
This section establishes the existence and uniqueness of equilibrium,
geometric convergence, and continuity of evolution under all admissible
transitions.

\subsection{Theorem 1: Contraction Mapping}

Let the corrective operator $\vec{E}$ satisfy
\[
    \|\vec{E}(\vec{\Delta}_a) - \vec{E}(\vec{\Delta}_b)\|
    \leq
    k \, \|\vec{\Delta}_a - \vec{\Delta}_b\|,
\qquad
    0 < k < 1,
\]
for all admissible deviation vectors $\vec{\Delta}_a, \vec{\Delta}_b$.
Then the deviation evolution
\[
    \vec{\Delta}_{t+1} = \vec{E}(\vec{\Delta}_t)
\]
forms a contraction mapping on $\mathcal{D} \subset \mathbb{R}^5$.

\paragraph{Proof.}
Immediate from the definition of a contraction operator. $\square$


\subsection{Theorem 2: Existence and Uniqueness of Equilibrium}

Under the contraction property of $\vec{E}$, there exists a unique
equilibrium deviation $\vec{\Delta}^\ast$ satisfying
\[
    \vec{E}(\vec{\Delta}^\ast) = \vec{\Delta}^\ast.
\]

\paragraph{Proof.}
By Banach's Fixed Point Theorem, every contraction on a closed, bounded,
complete space admits a unique fixed point. $\square$


\subsection{Theorem 3: Global Convergence}

For any initial deviation vector $\vec{\Delta}_0$, the deviation sequence
\[
    \vec{\Delta}_0, \vec{\Delta}_1, \vec{\Delta}_2, \dots
\]
converges to the unique equilibrium $\vec{\Delta}^\ast$ according to
\[
    \|\vec{\Delta}_t - \vec{\Delta}^\ast\|
    \leq
    k^t \, \|\vec{\Delta}_0 - \vec{\Delta}^\ast\|.
\]

\paragraph{Proof.}
Follows from repeated application of the contraction inequality. $\square$


\subsection{Theorem 4: Stability of Equilibrium}

The equilibrium deviation $\vec{\Delta}^\ast$ is globally stable:

\[
    \forall \, \epsilon > 0 \; \exists \, T :
    t > T \;\Rightarrow\;
    \|\vec{\Delta}_t - \vec{\Delta}^\ast\| < \epsilon.
\]

\paragraph{Proof.}
Direct consequence of geometric convergence established in Theorem~3. $\square$


\subsection{Theorem 5: No Oscillation}

Deviation trajectories cannot oscillate:
\[
    \vec{\Delta}_{t+1} - \vec{\Delta}_{t}
\]
always has a consistent sign relative to movement toward $\vec{\Delta}^\ast$.

\paragraph{Proof.}
Since the operator $\vec{E}$ is contracting and monotonic with respect to
$F$, it cannot produce a deviation that increases distance to equilibrium.
Thus reversal is impossible. $\square$


\subsection{Theorem 6: Continuity of Structural Motion}

Let $C_t$ be the structural correction yielding the next state:
\[
    X_{t+1} = X_t + C_t.
\]
If $\|C_t\| \leq L$ for all $t$, then the trajectory in state space is
continuous and contains no discontinuous transitions or heuristic jumps.

\paragraph{Proof.}
The update rule consists of a sum with a bounded term; hence the motion is
continuous in the norm topology of the state space. $\square$


\subsection{Theorem 7: Boundedness of FXI}

Under the boundedness of $\vec{\Delta}$ and continuity of $F$, the
equilibrium indicator satisfies:
\[
    \mathrm{FXI}( \vec{\Delta}_t ) \leq M,
\]
for some constant $M > 1$, for all $t$.

\paragraph{Proof.}
Since $\vec{\Delta}_t$ remains in a compact subset of $\mathbb{R}^5$ and
$F$ is continuous, its image is bounded. $\square$


\subsection{Theorem 8: Forward Completeness}

The system evolution is forward-complete: for every initial state
$X_0$, there exists a unique infinite trajectory
\[
    X_0, X_1, X_2, \dots
\]
compatible with the deviation evolution and bounded corrections.

\paragraph{Proof.}
From the existence of a unique deviation trajectory and bounded
corrections, the corresponding state trajectory is uniquely determined. $\square$


\subsection{Theorem 9: Structural Invariance of Equilibrium}

If $X^\ast$ is a state such that
\[
    D(X^\ast) = \vec{\Delta}^\ast,
\]
then the update rule leaves $X^\ast$ invariant:
\[
    X_{t+1} = X_t = X^\ast.
\]

\paragraph{Proof.}
Since $\vec{\Delta}^\ast$ is the fixed point of $\vec{E}$, the admissible
correction is zero; thus the system performs no structural movement. $\square$

\section{Critical Scenarios}

Critical scenarios represent extreme system configurations in which one
or more deviation components approach their admissible limits. FRE
Version~2.0 must guarantee stability, boundedness, and continuity even
in such conditions. This section defines the behavior of the deviation
vector, the equilibrium indicator, and the corrective operator under
structural stress.

\subsection{Boundary States}

A boundary state occurs when at least one deviation component saturates:
\[
    |\Delta_i| \rightarrow \Delta_i^{\max},
\qquad
    i \in \{m, L, H, R, C\}.
\]

Despite approaching structural limits, the system must satisfy:

\begin{itemize}
    \item bounded deviation evolution,
    \item bounded FXI,
    \item existence of an admissible correction,
    \item no discontinuous adjustments,
    \item continued movement toward equilibrium.
\end{itemize}

The corrective operator cannot diverge or change sign in boundary
conditions.

\subsection{Critical Expansion}

Critical expansion is defined as a state where:
\[
    F(\vec{\Delta}) \gg 1,
\]
indicating severe structural overextension (e.g., exposure overload,
insufficient liquidity, margin degradation).

FRE behavior:

\begin{itemize}
    \item $\vec{E}(\vec{\Delta})$ produces a strong contraction,
    \item deviation decreases monotonically,
    \item FXI moves downward but remains continuous,
    \item structural corrections $C_t$ remain bounded.
\end{itemize}

The system cannot experience a jump to a singular state or a
discontinuous liquidation.

\subsection{Critical Compression}

Critical compression is the opposite extreme:
\[
    F(\vec{\Delta}) \ll 1,
\]
reflecting excessive conservatism, over-hedging, or over-collateralization.

FRE behavior:

\begin{itemize}
    \item $\vec{E}(\vec{\Delta})$ expands deviation,
    \item movement toward equilibrium is monotonic,
    \item no oscillation or reversal is possible,
    \item contraction geometry is preserved.
\end{itemize}

The system remains structurally viable even in over-conservative
configurations.

\subsection{Multi-Component Stress}

A multi-component stress occurs when several deviation components
simultaneously approach their extremes:
\[
    \vec{\Delta}
    \approx
    (\Delta_m^{\max},
     \Delta_L^{\max},
     \Delta_H^{\max},
     \Delta_R^{\max},
     \Delta_C^{\max}).
\]

In this case, FRE guarantees:

\begin{itemize}
    \item all components move toward equilibrium concurrently,
    \item the corrective operator remains bounded in every dimension,
    \item no component may reverse direction,
    \item contraction in $\mathbb{R}^5$ remains strict.
\end{itemize}

This property is essential for complex risk–engine environments such as
decentralized exchanges or clearing systems.

\subsection{Edge-of-Domain Behavior}

Let $\mathcal{D}$ denote the admissible domain of $\vec{\Delta}$. At the
boundary $\partial \mathcal{D}$, FRE must satisfy:

\[
    \vec{E}(\vec{\Delta}) \in \mathcal{D},
\qquad
    \forall \vec{\Delta} \in \partial \mathcal{D}.
\]

Thus:

\begin{itemize}
    \item deviations cannot leave the admissible domain,
    \item admissible corrections always exist,
    \item the update rule is well-defined everywhere.
\end{itemize}

This guarantees that the system is globally well-posed.

\subsection{Critical FXI Conditions}

Extreme FXI values represent system-wide imbalance. FRE stipulates:

\[
    F(\vec{\Delta}) \rightarrow M
    \;\Rightarrow\;
    \vec{E}(\vec{\Delta}) - \vec{\Delta}
    \text{ is maximally contracting},
\]

\[
    F(\vec{\Delta}) \rightarrow 0
    \;\Rightarrow\;
    \vec{E}(\vec{\Delta}) - \vec{\Delta}
    \text{ is maximally expanding}.
\]

Thus, the structural correction strength increases near the limits of the
admissible domain.

\subsection{No Failure Modes}

FRE Version~2.0 has no internal failure modes in the deviation evolution:

\begin{itemize}
    \item no divergence,
    \item no unbounded paths,
    \item no oscillations,
    \item no structural dead zones,
    \item no discontinuous jumps.
\end{itemize}

Every deviation trajectory is guaranteed to converge to equilibrium or
remain in equilibrium.

\section{Simulation Framework}

The simulation framework provides a computational environment for
evaluating the dynamics, stability properties, stress behavior, and
trajectory geometry of the Flexionization Risk Engine (FRE) Version~2.0.
The simulator implements the formal deviation evolution, corrective
operator, structural bounds, and equilibrium analysis defined in previous
sections.

\subsection{Simulation Inputs}

A simulation run requires the following inputs:

\begin{itemize}
    \item initial system state $X_0$,
    \item deviation operator $D$,
    \item corrective operator $\vec{E}$,
    \item equilibrium indicator $F$,
    \item simulation horizon $T \in \mathbb{N}$,
    \item admissible correction bound $L$,
    \item domain constraints $\mathcal{D} \subset \mathbb{R}^5$.
\end{itemize}

These elements uniquely determine the structural evolution of the system
under the FRE 2.0 dynamics.

\subsection{Core Evolution Loop}

The simulation uses a discrete-time update cycle:

\[
    \vec{\Delta}_t = D(X_t),
\qquad
    \vec{\Delta}_{t+1} = \vec{E}(\vec{\Delta}_t),
\qquad
    X_{t+1} = X_t + C_t,
\]

with the correction vector $C_t$ chosen such that:

\[
    D(X_t + C_t) = \vec{E}(D(X_t)).
\]

The correction must satisfy the boundedness constraint:
\[
    \|C_t\| \leq L.
\]

\subsection{Trajectory Recording}

For each time step $t$, the simulator records:

\begin{itemize}
    \item system state $X_t$,
    \item deviation vector $\vec{\Delta}_t$,
    \item equilibrium indicator $F(\vec{\Delta}_t)$,
    \item corrective movement $C_t$,
    \item distance to equilibrium
    \[
        d_t = \|\vec{\Delta}_t - \vec{\Delta}^\ast\|.
    \]
\end{itemize}

This forms the full structural trajectory of the system.

\subsection{Stability Measurements}

The simulator computes several stability metrics:

\begin{itemize}
    \item \textbf{contraction ratio}
    \[
        k_t
        =
        \frac{
            \|\vec{\Delta}_{t+1} - \vec{\Delta}^\ast\|
        }{
            \|\vec{\Delta}_{t} - \vec{\Delta}^\ast\|
        },
    \]
    \item \textbf{FXI trajectory}
    \[
        F(\vec{\Delta}_0),
        F(\vec{\Delta}_1),
        \dots,
        F(\vec{\Delta}_T),
    \]
    \item \textbf{velocity of structural movement}
    \[
        v_t
        =
        \|\vec{\Delta}_{t+1} - \vec{\Delta}_t\|,
    \]
    \item \textbf{projection onto equilibrium gradient}.
\end{itemize}

These metrics verify the contraction and monotonicity properties of FRE.

\subsection{Stress Testing}

The simulator must support structured stress-testing scenarios:

\begin{itemize}
    \item single-component stress ($\Delta_i$ near maximum),
    \item multi-component stress,
    \item FXI extremes,
    \item domain-edge trajectories,
    \item sudden structural imbalance injections.
\end{itemize}

Stress tests verify the robustness properties defined in Section~6.

\subsection{Visualization Requirements}

The simulator should provide:

\begin{itemize}
    \item deviation trajectories in $\mathbb{R}^5$,
    \item FXI curve over time,
    \item contraction profile,
    \item radial convergence plot toward equilibrium,
    \item stress-test overlays,
    \item 2D/3D projections of deviation geometry.
\end{itemize}

Visualization clarifies the stability geometry of FRE.

\subsection{Admissibility Validation}

At each time step the simulator validates:

\begin{itemize}
    \item $\vec{\Delta}_t \in \mathcal{D}$,
    \item boundedness of corrections,
    \item monotonic motion toward equilibrium,
    \item absence of direction reversal,
    \item continuity of structural movement.
\end{itemize}

A warning is raised if any condition is violated.

\subsection{Termination Conditions}

A simulation may terminate early if:

\begin{itemize}
    \item equilibrium is reached,
    \item deviation norm falls below tolerance
    \[
        \|\vec{\Delta}_t - \vec{\Delta}^\ast\| < \epsilon,
    \]
    \item structural constraints are violated,
    \item corrective operator becomes undefined (invalid input domain).
\end{itemize}

\subsection{Output}

Simulation outputs include:

\begin{itemize}
    \item full trajectory $\{X_t\}$,
    \item deviation trajectory $\{\vec{\Delta}_t\}$,
    \item FXI sequence,
    \item convergence diagnostics,
    \item stability metrics,
    \item stress-test results,
    \item visualization files (if enabled).
\end{itemize}

These outputs are essential for validating FRE 2.0 implementation and
evaluating real-world risk behavior.

\section{Conclusion}

The Flexionization Risk Engine (FRE) Version~2.0 establishes a fully
structural, multidimensional, and mathematically rigorous framework for
risk control across financial and computational architectures. Unlike
traditional systems that rely on volatility triggers, price movements, or
heuristic liquidation thresholds, FRE derives all corrective behavior
from internal system dynamics.

By introducing a five-dimensional deviation vector
\[
    \vec{\Delta} = (\Delta_m, \Delta_L, \Delta_H, \Delta_R, \Delta_C),
\]
an equilibrium indicator \( \mathrm{FXI} \), and a vector corrective
operator \( \vec{E} \), FRE provides:

\begin{itemize}
    \item continuous and bounded structural evolution,
    \item geometric contraction toward equilibrium,
    \item global stability guarantees,
    \item resilience under critical and boundary scenarios,
    \item deterministic and fully predictable corrective behavior.
\end{itemize}

The stability theorems ensure that every admissible trajectory converges
to a unique equilibrium deviation \( \vec{\Delta}^\ast \), while the
simulation framework provides a practical means to evaluate dynamics,
stress conditions, and long-term convergence in real-world systems.

FRE Version~2.0 therefore serves as a unified structural foundation for:

\begin{itemize}
    \item CeFi risk engines,
    \item DeFi protocol safety modules,
    \item automated hedging systems,
    \item liquidity and leverage control architectures,
    \item banking and clearing systems,
    \item next-generation on-chain risk infrastructures.
\end{itemize}

The specification presented in this document provides the complete
mathematical basis for implementing, analyzing, and extending the
Flexionization Risk Engine. It forms the theoretical core for forthcoming
developments, including FRE Version~2.1 (Matrix Interaction Model),
advanced simulation tools, and integration into the NGT ecosystem.

FRE 2.0 represents a structural shift in risk–control design: the system
no longer reacts to external volatility, but instead continuously
corrects itself based on internal equilibrium geometry. This establishes
a new class of risk engines—deterministic, stable, and structurally
self-correcting—capable of maintaining equilibrium under all admissible
conditions.

\begin{thebibliography}{99}

\bibitem{Bogdanov2025Theory}
Bogdanov, M.
\textit{Flexionization: Formal Theory of Dynamic Quantitative Equilibrium}.
Version 1.5, 2025. Zenodo.
DOI: 10.5281/zenodo.17618947.

\bibitem{Bogdanov2025Immune}
Bogdanov, M.
\textit{Flexion-Immune Model}.
Version 1.1, 2025. Zenodo.
DOI: 10.5281/zenodo.17624206.

\bibitem{Bogdanov2025FRE1.1}
Bogdanov, M.
\textit{Flexionization Risk Engine (FRE) Version 1.1}.
2025. Zenodo.
DOI: 10.5281/zenodo.17628118.

\bibitem{Strogatz}
Strogatz, S.
\textit{Nonlinear Dynamics and Chaos}.
Westview Press.

\bibitem{Khalil}
Khalil, H.
\textit{Nonlinear Systems}.
Prentice Hall.

\bibitem{Bertsekas}
Bertsekas, D.
\textit{Dynamic Programming and Optimal Control}.
Athena Scientific.

\bibitem{Rockafellar}
Rockafellar, R.
\textit{Convex Analysis}.
Princeton University Press.

\bibitem{Banach1922}
Banach, S.
``Sur les opérations dans les ensembles abstraits''.
\textit{Fundamenta Mathematicae}, 1922.

\end{thebibliography}

\end{document}